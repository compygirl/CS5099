\let\textcircled=\pgftextcircled
\chapter{Introduction}
\label{chap:intro}
%----------------------------------------------------------------------------------------
%   INTRO
%----------------------------------------------------------------------------------------
\initial{C}onstraint Programming (CP) is very powerful tool that is being increasingly used in academia and industry for solving combinatorial problems which are hard to express mathematically. It is widely researched in Computer Science, Artificial Intelligence and operational research \cite{hanbook_cp}. Constraint modelling is every important and at the same is hard \cite{savile_row_paper}. Puzzle games are very convenient in representation of \textbf{Constraint Satisfaction Problems}. By solving the problems using puzzle games the obtained knowledge could be applied to the other relevant to the problem areas such as traffic engineering \cite{industry,nonogram_reinforcement}.


There are so many works and researches was dedicated to the puzzle games such as Sudoku, but not many related to the Nonogram \cite{Sudoku_as_CP}. The Nomogram was chosen as another example of the combinatorial problems, which need to be automatically generated by using constraint programming approaches. The initial step was to create a solver using constraints of the puzzle. For the implementation of the solver it was used Essence’ constraint programming language. Then by using this constraint model (i.e. solver) create a system which will generate instances of the Nonogram. Solvable instances are the ones which has only one valid solution. Randomly generated instances could have more than one solution or could take too long to be solved. That happens because of several reasons which will be discovered and analysed. Based on these obtained knowledge it could become more clear to come up with some algorithms which will be used for generation of different levels of the puzzle. 



%----------------------------------------------------------------------------------------
%  SECTION : CONSTRAINT PROGRAMMING
%----------------------------------------------------------------------------------------
\section{Constraint Programming}
\label{sec:subsec_cp}

Constraint programming is the way of programming where instead of doing a continuous sequence of task to be executed, it requires to write the properties of the problem, which is not need to be defined in particular order. The main idea of the Constraint Programming is to declare \textbf{variables}, define set of possible values for each variable (each set of values for the variable is called \textbf{domain}) and specify a relations between the variables, which are called \textbf{constraints}. \textbf{Figure} shows simple example of the Constraints\cite{hanbook_cp,nonogram_good,}.

%----------------------------------------------------------------------------------------
%   EXAMPLE 1: 
%----------------------------------------------------------------------------------------
\begin{addmargin}[4em]{1em}
\textbf{Example 1:}
\label{ex:example1}
\end{addmargin}
% \hfill
\begin{table}[h]
    \begin{center}
    \begin{tabular}{ c c c }
         \hline
         Variables & Domains & Constraints \\ 
         \hline

         $X_{1}$ & $X_{1}\in \left\{ 1,2,3\right\}$ & $X_{1}>X_{2}$ \\  
         $X_{2}$ & $X_{2}\in \left\{ 1,2,3\right\}$ & $X_{1}>X_{3}$ \\
         $X_{3}$ & $X_{3}\in \left\{ 2,3\right\}$ & $X_{2}\neq X_{3}$ 
    \end{tabular}
    \caption{Declaration of Variables, Domains and Constraints.}
    \label{table:cp}
    \end{center}
\end{table}



\begin{addmargin}[4em]{1em}
After applying constraints on the variables, it will remove the impossible values from their domains:
\end{addmargin}





% \begin{table}
% % \begin{minipage}[h]{\dimexpr\textwidth-2cm}
%     \begin{minipage}{0.4\textwidth}
%             \setlength{\unitlength}{0.75cm}
%             \begin{picture}(12,4)
%             \thicklines
%             % points:
%             \put(4,3){\circle*{0.1}}
%             \put(3.3,2.9){$X_{2}$}
%             \put(3,1){\circle*{0.1}}
%             \put(2.7,0.5){$X_{1}$}
%             \put(5,1){\circle*{0.1}}
%             \put(4.7,0.5){$X_{3}$}
%             % lines:
%             \put(3,1){\vector(1,0){2}}
%             \put(3,1){\vector(1,2){1}}
%             \put(5,1){\vector(-1,2){1}}
%             %  sings:
%             \put(3.1,1.8){$>$}
%             \put(3.9,0.7){$>$}
%             \put(4.6,1.8){$\neq$}
%             \end{picture}
%     \end{minipage}
%     ~
%     \begin{minipage}{0.4\textwidth}
%           \begin{tabular}{ c c }
%                 \hline
%                 Value  elimination& Value assignment\\ 
%                 \hline

%                 $X_{1}\in \left\{ \cancel{1},\cancel{2},3\right\}$ & $X_{1}=3$ \\  
%                 $X_{2}\in \left\{ 1,\cancel{2},\cancel{3}\right\}$ & $X_{2}=1$ \\
%                 $X_{3}\in \left\{ 2,\cancel{3}\right\}$ & $X_{3}=2$
%             \end{tabular}
%     \end{minipage}\\[1.5cm]
%     % \captionof{table}{Simple example of the Constraint Satisfaction Problem}
% % \end{minipage}
% \end{table}

\begin{center}
% \begin{addmargin}[4em]{1em}
% \begin{table}[h]
    \begin{minipage}[b]{0.4\linewidth}
        \centering
        % \captionsetup{width=.6\linewidth}
            \setlength{\unitlength}{0.75cm}
            \begin{picture}(12,4)
            \thicklines
            % points:
            \put(4,3){\circle*{0.1}}
            \put(3.3,2.9){$X_{2}$}
            \put(3,1){\circle*{0.1}}
            \put(2.7,0.5){$X_{1}$}
            \put(5,1){\circle*{0.1}}
            \put(4.7,0.5){$X_{3}$}
            % lines:
            \put(3,1){\vector(1,0){2}}
            \put(3,1){\vector(1,2){1}}
            \put(5,1){\vector(-1,2){1}}
            %  sings:
            \put(3.1,1.8){$>$}
            \put(3.9,0.7){$>$}
            \put(4.6,1.8){$\neq$}
            \end{picture}
        % \caption{Graphic representation of constraints}
        \captionof{figure}{Graphic representation}
        \label{triangle_cp}
    \end{minipage}
    ~
    \begin{minipage}[b]{0.4\linewidth}

       \begin{tabular}{ c c }
            \hline
            Value  elimination& Value assignment\\ 
            \hline

            $X_{1}\in \left\{ \cancel{1},\cancel{2},3\right\}$ & $X_{1}=3$ \\  
            $X_{2}\in \left\{ 1,\cancel{2},\cancel{3}\right\}$ & $X_{2}=1$ \\
            $X_{3}\in \left\{ 2,\cancel{3}\right\}$ & $X_{3}=2$
        \end{tabular}
        \captionof{table}{Assignment of variables}
        \label{fig:image}
    \end{minipage}
% \end{table}
% \end{addmargin}
\end{center}


%----------------------------------------------------------------------------------------
%  SECTION : NONOGRAM
%----------------------------------------------------------------------------------------
\section{Nonogram}
\label{sec:subsec_nonogram}
As puzzle games can be used as good example of constraint satisfaction problems (CSP) there are many researches related to the puzzle games such as Sudoku, N queens problem and Nonogram \cite{Sudoku_as_CP}. 

Nonogram was invented in 1980s in Japan and was called as “Hanjie” \cite{nonogram_reinforcement}. Nonogram also known as Picross or Griddlers in different parts of the world \cite{nonogram_reinforcement,nonogram_good}. The classic version of the game is black and white version, where nonogram players must colour some cells of the $m\times n$ size grid to the black or leave them white accordingly a given numbers in the two sided matrices. The numbers and their orders in these matrices specifies the amount of continuously coloured cells in one block in the any given row or column. Each block should be separated by at least one cell.
The main goal of the game is to complete the colouring of the required cells of the grid by satisfying each number (called \textbf{clue}) on the sided matrices, and revealing the hidden image \cite{nonogram_good}. The Nonogram could be a colourful as well and then each clue specifies not just the amount of continuously coloured cells but also the colour of the block. For the this dissertation it was decided to use a classic one (i.e. black and white).

The detailed information about the rules and definitions of the game will be discussed in the design and implementation chapter.



%----------------------------------------------------------------------------------------
%   SUBSECTION: 
%----------------------------------------------------------------------------------------
\section{Project Ojectives}
\label{subsec:objectives}
This project is the individual research work about the development of the puzzle game solver, called Nonogram, based on the concept of the Constraint Satisfaction Problem (CSP).
The entire project was initilly outlined into the following three parts.

\subsection{Primary Ojectives}
\label{subsec:primary_obj}
\begin{enumerate}[(a)]
    \item The main objective of the project is to automatically generate puzzles. The implemented software should be done based on the constraints of the game and solve any levels of the provided instances of the game. The instances should be provided by the text file.
    \item Write a system which can generate random instances of the puzzle, which are valid but may not have an answer. Use the system from (a) to check if these instances are solvable.
\end{enumerate}

\subsection{Secondary Ojectives}
\label{subsec:secondary_obj}
\begin{enumerate}[(a)]
    \item Write a guided system which generates instances of the puzzle, and tries to make solvable instances. This will recursively call the solver from part (a), and be guided towards correct and solvable instances.
    \item Implement the game itself, which will use the system from 2.2(a) and 2.1(a) to produce problems and check they are solvable.
\end{enumerate}


\subsection{Tertiary Ojectives}
\label{subsec:tertiary_obj}
\begin{enumerate}[(a)]
    \item Investigate how to generate harder levels, and both create and solve levels of the puzzle faster.
    \item Compare how the computer rates the difficulty of levels with how human players rate difficulty of levels.
\end{enumerate}

