%\title{Dissertation: Nonogram in CP}
\documentclass[oneside, a4paper,11pt,leqno,openbib]{memoir}
%
% Declare figure/table as a subfloat.
\newsubfloat{figure}
\newsubfloat{table}
% Better page layout for A4 paper, see memoir manual.
\settrimmedsize{297mm}{210mm}{*}
\setlength{\trimtop}{0pt} 
\setlength{\trimedge}{\stockwidth} 
\addtolength{\trimedge}{-\paperwidth} 
\settypeblocksize{634pt}{448.13pt}{*} 
\setulmargins{4cm}{*}{*} 
\setlrmargins{*}{*}{1.5} 
\setmarginnotes{17pt}{51pt}{\onelineskip} 
\setheadfoot{\onelineskip}{2\onelineskip} 
\setheaderspaces{*}{2\onelineskip}{*} 
\checkandfixthelayout
%
\frenchspacing
% Font with math support: New Century Schoolbook
\usepackage{fouriernc}
\usepackage[T1]{fontenc}
\usepackage{xr}
\usepackage{lipsum}       % for sample text
\usepackage{tabu}
% \usepackage{enumitem}

% \newcommand{\strike}[1]{#1} % Some LaTeX wizardry here
% \renewcommand{\arraystretch}{2}

% Note: This is automatically set by memoir class. Nevertheless \OnehalfSpacing 
% enables double spacing but leaves single spaced for captions for instance. 
\OnehalfSpacing 
%
% Sets numbering division level
\setsecnumdepth{subsection} 
\maxsecnumdepth{subsubsection}
%
% Chapter style (taken and slightly modified from Lars Madsen Memoir Chapter 
% Styles document
\usepackage{calc, soul, fourier}
\makeatletter 
\newlength\dlf@normtxtw 
\setlength\dlf@normtxtw{\textwidth} 
\newsavebox{\feline@chapter} 
\newcommand\feline@chapter@marker[1][4cm]{%
	\sbox\feline@chapter{% 
		\resizebox{!}{#1}{\fboxsep=1pt%
			\colorbox{gray}{\color{white}\thechapter}% 
		}}%
		\rotatebox{90}{% 
			\resizebox{%
				\heightof{\usebox{\feline@chapter}}+\depthof{\usebox{\feline@chapter}}}% 
			{!}{\scshape\so\@chapapp}}\quad%
		\raisebox{\depthof{\usebox{\feline@chapter}}}{\usebox{\feline@chapter}}%
} 
\newcommand\feline@chm[1][4cm]{%
	\sbox\feline@chapter{\feline@chapter@marker[#1]}% 
	\makebox[0pt][c]{% aka \rlap
		\makebox[1cm][r]{\usebox\feline@chapter}%
	}}
\makechapterstyle{daleifmodif}{
	\renewcommand\chapnamefont{\normalfont\Large\scshape\raggedleft\so} 
	\renewcommand\chaptitlefont{\normalfont\Large\bfseries\scshape} 
	\renewcommand\chapternamenum{} \renewcommand\printchaptername{} 
	\renewcommand\printchapternum{\null\hfill\feline@chm[2.5cm]\par} 
	\renewcommand\afterchapternum{\par\vskip\midchapskip} 
	\renewcommand\printchaptertitle[1]{\color{gray}\chaptitlefont\raggedleft ##1\par}
} 
\makeatother 
\chapterstyle{daleifmodif}
%
% UoB guidelines:
%
% The pages should be numbered consecutively at the bottom centre of the
% page.
\makepagestyle{myvf} 
\makeoddfoot{myvf}{}{\thepage}{} 
\makeevenfoot{myvf}{}{\thepage}{} 
\makeheadrule{myvf}{\textwidth}{\normalrulethickness} 
\makeevenhead{myvf}{\small\textsc{\leftmark}}{}{} 
\makeoddhead{myvf}{}{}{\small\textsc{\rightmark}}
\pagestyle{myvf}
%
% Creates indexes for Table of Contents, List of Figures, List of Tables and Index
\makeindex
\usepackage{import}
% Add other packages needed for chapters here. For example:
\usepackage{amsfonts} 					%Calls Amer. Math. Soc. (AMS) fonts
\usepackage[centertags]{amsmath}			%Writes maths centred down
\usepackage[makeroom]{cancel}
\usepackage{stmaryrd}					%New AMS symbols
\usepackage{amssymb}					%Calls AMS symbols
\usepackage{amsthm}						%Calls AMS theorem environment
\usepackage{newlfont}					%Helpful package for fonts and symbols
\usepackage{graphicx}					%Calls figure environment
\graphicspath{ {img/} }
\usepackage{longtable, rotating}			%Long tab environments including rotation. 
\usepackage{colortbl}					%Makes coloured tables
\usepackage{wasysym}					%More math symbols
\usepackage{listings}
\usepackage{mathrsfs} %Even more math symbols
\usepackage{mathtools}
\usepackage{float}						%Helps to place figures, tables, etc. 
\usepackage{verbatim}					%Permits pre-formated text insertion
\usepackage{upgreek}					%Calls other kind of greek alphabet
\usepackage{latexsym}                   %Extra symbols
\usepackage{pdfpages}
\usepackage[square,numbers, sort&compress]{natbib}		%Calls bibliography commands 
\usepackage{url}						%Supports url commands
\usepackage[english]{babel}		%For languages characters and hyphenation
\usepackage{color}                    				%Creates coloured text and background
\usepackage[colorlinks=true, allcolors=black]{hyperref}
\usepackage{memhfixc}					%Must be used on memoir document 
\usepackage{enumerate}					%For enumeration counter
\usepackage{footnote}					%For footnotes
% \usepackage{microtype}					%Makes pdf look better.
\usepackage{rotfloat}					%For rotating and float environments as tables, 
\usepackage{alltt}						%LaTeX commands are not disabled in 
\usepackage{tikz}						%geometric/algebraic description.
\usepackage{booktabs}
% \usepackage{fontspec}
\usepackage{awesomebox}
\usepackage{multicol}
% \usepackage{caption}
\usepackage{scrextend}
\usepackage{subcaption}
\usepackage{float}
\usepackage{array}
\usepackage{makecell}
% \usepackage{seqsplit}
\usepackage{tabularx}
\restylefloat{table}

\newcommand*{\tikzbullet}[2]{%
   \setbox0=\hbox{\strut}%
   \begin{tikzpicture}
     \useasboundingbox (-.15em,0) rectangle (.15em,\ht0);
     \filldraw[draw=#1,fill=#2] (0,0.5\ht0) circle[radius=.15em];
   \end{tikzpicture}%
}
\newcolumntype{x}[1]{>{\centering\arraybackslash}p{#1}}

% \usepackage{array, caption, floatrow, tabularx, makecell, booktabs}%
% \usepackage{floatrow}
\setlength{\columnsep}{1cm}
\setlength{\parindent}{0em}
\setlength{\parskip}{1em}
\usetikzlibrary{arrows,shapes, automata,backgrounds, petri,topaths}				%To use diverse features from tikz		
%							
%Reduce widows  (the last line of a paragraph at the start of a page) and orphans 
% (the first line of paragraph at the end of a page)
\widowpenalty=1000
\clubpenalty=1000
%
% New command definitions for my thesis
%
\newcommand{\keywords}[1]{\par\noindent{\small{\textbf Keywords:} #1}} %Defines keywords small section
\newcommand{\parcial}[2]{\frac{\partial#1}{\partial#2}}                             %Defines a partial operator
\newcommand{\vectorr}[1]{\mathbf{#1}}	%Defines a bold vector
\newcommand{\vecol}[2]{\left
(										%Defines a column vector
	\begin{array}{c} 
		\displaystyle#1 \\
		\displaystyle#2
	\end{array}\right)}
\newcommand{\mados}[4]{\left(                                                                       %Defines a 2x2 matrix
	\begin{array}{cc}
		\displaystyle#1 &\displaystyle #2 \\
		\displaystyle#3 & \displaystyle#4
	\end{array}\right)}
\newcommand{\pgftextcircled}[1]{                                                                    %Defines encircled text
    \setbox0=\hbox{#1}%
    \dimen0\wd0%
    \divide\dimen0 by 2%
    \begin{tikzpicture}[baseline=(a.base)]%
        \useasboundingbox (-\the\dimen0,0pt) rectangle (\the\dimen0,1pt);
        \node[circle,draw,outer sep=0pt,inner sep=0.1ex] (a) {#1};
    \end{tikzpicture}
}
\newcommand{\range}[1]{\textnormal{range }#1}                                             %Defines range operator
\newcommand{\innerp}[2]{\left\langle#1,#2\right\rangle}                                 %Defines inner product
\newcommand{\prom}[1]{\left\langle#1\right\rangle}                                         %Defines average operator
\newcommand{\tra}[1]{\textnormal{tra} \: #1}                                                       %Defines trace operator
\newcommand{\sign}[1]{\textnormal{sign\,}#1}                                                   %Defines sign operator
\newcommand{\sech}[1]{\textnormal{sech} #1}                                                  %Defines sech
% \newcommand{\diag}[1]{\textnormal{diag} #1}                                                    %Defines diag operator
\newcommand{\arcsech}[1]{\textnormal{arcsech} #1}                                       %Defines arcsech
\newcommand{\arctanh}[1]{\textnormal{arctanh} #1}                                         %Defines arctanh
%Change tombstone symbol
\newcommand{\blackged}{\hfill$\blacksquare$}
\newcommand{\whiteged}{\hfill$\square$}
\newcommand\diag[4]{%
  \multicolumn{1}{p{#2}|}{\hskip-\tabcolsep
  $\vcenter{\begin{tikzpicture}[baseline=0,anchor=south west,inner sep=#1]
  \path[use as bounding box] (0,0) rectangle (#2+2\tabcolsep,\baselineskip);
  \node[minimum width={#2+2\tabcolsep},minimum height=\baselineskip+\extrarowheight] (box) {};
  \draw (box.south west) -- (box.north east);
  \node[anchor=south west] at (box.south west) {#3};
  \node[anchor=north east] at (box.north east) {#4};
 \end{tikzpicture}}$\hskip-\tabcolsep}}

\newcounter{proofcount}
\renewenvironment{proof}[1][\proofname.]{\par
 \ifnum \theproofcount>0 \pushQED{\whiteged} \else \pushQED{\blackged} \fi%
 \refstepcounter{proofcount}
 \normalfont 
 \trivlist
 \item[\hskip\labelsep
       % \itshape
   {\textbf\em #1}]\ignorespaces
}{%
 \addtocounter{proofcount}{-1}
 \popQED\endtrivlist
}

\DeclareMathOperator*{\argmax}{arg\,max}

%
% New definition of square root:
% it renames \sqrt as \oldsqrt
\let\oldsqrt\sqrt
% it defines the new \sqrt in terms of the old one
\def\sqrt{\mathpalette\DHLhksqrt}
\def\DHLhksqrt#1#2{%
\setbox0=\hbox{$#1\oldsqrt{#2\,}$}\dimen0=\ht0
\advance\dimen0-0.2\ht0
\setbox2=\hbox{\vrule height\ht0 depth -\dimen0}%
{\box0\lower0.4pt\box2}}
%
% My caption style
\newcommand{\mycaption}[2][\@empty]{
	% \captionnamefont{\scshape} 
	\changecaptionwidth
	\captionwidth{0.9\linewidth}
	\captiondelim{.\:} 
	\indentcaption{0.75cm}
	\captionstyle[\centering]{}
	\setlength{\belowcaptionskip}{10pt}
	\ifx \@empty#1 \caption{#2}\else \caption[#1]{#2}
}
%
% My subcaption style
\newcommand{\mysubcaption}[2][\@empty]{
	\subcaptionsize{\small}
	\hangsubcaption
	\subcaptionlabelfont{\rmfamily}
	\sidecapstyle{\raggedright}
	\setlength{\belowcaptionskip}{10pt}
	\ifx \@empty#1 \subcaption{#2}\else \subcaption[#1]{#2}
}
%
%An initial of the very first character of the content
\usepackage{lettrine}
\newcommand{\initial}[1]{%
	\lettrine[lines=3,lhang=0.33,nindent=0em]{
		\color{gray}
     		{\textsc{#1}}}{}}
%
% Theorem styles used in my thesis
%
\theoremstyle{plain}
\newtheorem{theo}{Theorem}[chapter]
\theoremstyle{plain}
\newtheorem{prop}{Proposition}[chapter]
\theoremstyle{plain}
\theoremstyle{definition}
\newtheorem{dfn}{Definition}[chapter]
\theoremstyle{plain}
\newtheorem{lema}{Lemma}[chapter]
\theoremstyle{plain}
\newtheorem{cor}{Corollary}[chapter]
\theoremstyle{plain}
\newtheorem{resu}{Result}[chapter]
%
% Hyphenation for some words
%
\hyphenation{res-pec-tively}
\hyphenation{mono-ti-ca-lly}
\hyphenation{hypo-the-sis}
\hyphenation{para-me-ters}
\hyphenation{sol-va-bi-li-ty}
%
%
\begin{document}
% 
\frontmatter
\pagenumbering{roman}
\begin{titlingpage}
\newcommand{\HRule}{\rule{\linewidth}{0.5mm}} 
\center % Center everything on the page
%----------------------------------------------------------------------------------------
%   TITLE SECTION
%----------------------------------------------------------------------------------------
\HRule \\[0.5cm]
{ \huge Automated creation of Nonogram puzzle}\\[0.4cm]
{ \huge game with Constraint Programming}\\[0.3cm] % Title of your document
\HRule \\[1.5cm]
 %----------------------------------------------------------------------------------------
%   LOGO SECTION
%----------------------------------------------------------------------------------------
\includegraphics[width= 0.4\textwidth]{UniversityStAndrews}\\[1cm] 
{ \huge University of St Andrews}\\[0.4cm] % Title of your document
\textsc{\Large School of Computer Science}\\[0.4cm] % Minor heading such as course title
\Large MSc Artificial Intelligence\\% Your name
%----------------------------------------------------------------------------------------
%   AUTHOR SECTION
%----------------------------------------------------------------------------------------
~\\~\\\begin{minipage}{0.4\textwidth}
    \begin{flushleft} \large
        \emph{Author:}\\
        Aigerim \textsc{Yessenbayeva}\\% Your name
        \emph{Student ID:} 
        \textsc{090017799}\\% Your name
    \end{flushleft}
    \end{minipage}
    ~
    \begin{minipage}{0.5\textwidth}
    \begin{flushright} \large
    \emph{Supervisor:} \\
    Dr. Christopher \textsc{Jefferson} % Supervisor's Name
    \end{flushright}
\end{minipage}\\[2cm]
%----------------------------------------------------------------------------------------
%   DATE SECTION
%----------------------------------------------------------------------------------------
{\large \today}\\[2cm] % Date, change the \today to a set date if you want to be precise
%----------------------------------------------------------------------------------------
\vfill % Fill the rest of the page with whitespace
\end{titlingpage}


\chapter*{Abstract}
\begin{SingleSpace}

\initial{T}his report discusses in detail all the stages of design, implementation and evaluation of models that perform sentiment analysis and topic modelling on publicly available social media. The sentiment analysis model attempts to define whether a tweet has a positive or negative `mood' and expresses this `mood' as a number. The topic model was initially supposed to detect whether a tweet talks about any environmental issues that are related to the human impact on nature in urban areas (namely the Grangemouth area). Later, however, the model was changed to detecting whether a tweet talked about the UK General Election planned for June 2017 or any other election. 

This report focuses on different aspects of accessing and processing social media data, as well as using it for document classification.

The goal of this report is to evaluate the models built for the document classification, the results given by these models, as well as to discuss any possible improvements and further work in the field.
\end{SingleSpace}
\chapter*{Declaration}
\begin{SingleSpace}
\begin{quote}
\initial{I} declare that the material submitted for assessment is my own work except where credit is explicitly given to others by citation or acknowledgement. This work was performed during the current academic year except where otherwise stated. The main text of this project report is 18,272 words long, including project specification and plan. In submitting this project report to the University of St Andrews, I give permission for it to be made available for use in accordance with the regulations of the University Library. I also give permission for the report to be made available on the Web, for this work to be used in research within the University of St Andrews, and for any software to be released on an open source basis. I retain the copyright in this work, and ownership of any resulting intellectual property.

\vspace{1.5cm}
\noindent
\end{quote}
\end{SingleSpace}
\clearpage
% 
\renewcommand{\contentsname}{Table of Contents}
\maxtocdepth{subsection}
\tableofcontents*
\addtocontents{toc}{\par\nobreak \mbox{}\hfill{\textbf Page}\par\nobreak}
\clearpage
%
\listoffigures
\addtocontents{lof}{\par\nobreak\textbf{{\scshape Figure} \hfill Page}\par\nobreak}
\clearpage
%
\mainmatter
%
\import{chapters/Introduction/}{1chap.tex}
\import{chapters/ContextSurvey/}{2chap.tex}
\import{chapters/Ethics/}{5chap.tex}
\import{chapters/DesignImplementation/}{6chap.tex}
\import{chapters/UserStudy/}{7chap.tex}
\import{chapters/Evaluation/}{8chap.tex}
% \import{chapters/Conclusion/}{9chap.tex}
%
%
\appendix
\import{chapters/appendices/}{manual.tex}
% \import{chapters/appendices/}{keywords.tex}
% \import{chapters/appendices/}{requirements.tex}
%
\backmatter
\refstepcounter{chapter}
\bibliographystyle{unsrt}
\bibliography{bibliography.bib}
%   
\end{document}