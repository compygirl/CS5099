\let\textcircled=\pgftextcircled
\chapter{Ethics}
\label{chap:5}
%----------------------------------------------------------------------------------------
%   INTRO
%----------------------------------------------------------------------------------------
\initial{T}he project had a tertiary objective that was about about the comparison of how human players rate the levels of Nonogram against the rates made by computer. In order to conduct this study it was needed to involve some group of people, who would volunteer to participate in this study. The most of the participants were involved from the School of Computer Science in University of St Andrews. Essentially the idea of the experiment was to ask participants to solve the instances of the Nonogram. The players were provided with the instruction of the study as well as with the rules of the Nonogram. The users were provided with the same instances of the game in the same order with the same maximum limit of time to solve these instances. Each noted noted interval of time to solve one instance was the main purpose of the experiment. All volunteers were provided with the right to stop the experiment at any time when they wish. All users were kept anonymised. At the end of the experiment each participant was only asked about the rates of the puzzle levels and about the previous experience in this game. It was not that hard to involve the participants since the experiment was not involving any sensitive and private information that need to be analysed or unclosed. The only needed information from this study was the average time spent to solve particular type of instances by all participants. 

To conduct this study it was necessary to sing an approval form from the Ethical committee of the School of Computer Science in the University of St Andrews. 
